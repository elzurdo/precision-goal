Sequential hypothesis testing enables continuous data collection and stopping
when there's enough evidence to make a decision, rather than predetermining a sample
size.
Common approaches rely on accept/reject decision criteria as stopping rules,
which can introduce {\it confirmation bias} due to an unrepresentative sequence which is
likely in a small sample.
While this issue is often associated with the Frequentist {\it p-value} heuristic,
it also occurs in popular Bayesian metrics.
To address this, a predetermined parameter accuracy has been proposed
as a stopping criterion, separate from the decision criterion.
By demonstrating on dichotomous sampling \cite{kruschke2015doing} showed that defining in advance the
posterior precision as a stopping criterion eliminates the confirmation bias.
While this {\it Precision is the Goal} method is unbiassed,
practical challenges arise when the posterior distribution does not clearly indicate
whether the null hypothesis should be accepted or rejected—leading to an {\it inconclusive}
outcome.
To mitigate this, we propose a conservative variant called
{\it Enhanced Precision Is The Goal} which requires simultaneously meeting both the
precision objective and the decision criterion.
We tested this approach on synthetic data and found it significantly reduces
inconclusiveness. E.g, when testing for a fair coin the original method may result in 61\%
inconclusive outcomes where our variant just 1.6\%—at the cost of a moderate increase in sample
size (approximately 22\% larger).
We also provide an online calculator and source code to facilitate the adoption of the
approach in data collection planning and experimental analysis.


\
\

\cite{kruschke2015doing} introduced the "Precision is the Goal" method,
which eliminates bias in dichotomous outcome sampling by setting an objective precision threshold as the stopping rule.

We find in cases where the null hypothesis is true,
applying the "Precision is the Goal" method avoids false positives but results in a high rate of inconclusive decisions.


\
\


Sequential hypothesis testing is a process in which a practitioner does not predefine a
sample size but relies on assumptions about incoming data to indicate that enough has
been collected to make a decision about a hypothesis. Most commonly used approaches
apply an accept/reject decision criterion as a stopping criterion which makes results
prone to biases, such as confirmation bias due to the possibility of obtaining extreme
results. Although known as a problem with the frequentist {\it p-value} mechanism, it is not
magically resolved with Bayesian approaches. To resolve this issue, a predetermined
accuracy in parameter estimation has been advocated as a stopping criterion.
In particular \cite{kruschke2015doing} showed that by defining in advance an  posterior precision as
a stopping criterion, a method called {\it Precision is the Goal}, eliminates bias in the
case of dichotomous outcomes sampling. Whereas we find this to be true, we highlight
important practical details, in particular when the posterior doesn’t clearly indicate
if the null hypothesis may neither be rejected or accepted, i.e, a situation of
“inconclusiveness”. In cases where the null hypothesis is true, when applying
“Precision is the Goal” we find that even though the null hypothesis is never
incorrectly rejected (free from False positives), the majority of the decisions may
be inconclusive. 
To address this problem we introduce a conservative variant of this method which we call
{\it Enhanced Precision Is The Goal} in which the stopping criterion simultaneously
requires fulfilling the precision objective as well as the
reject/accept decision criterions. We apply to synthetic  data and show that it may
substantially reduce inconclusiveness at the cost of larger sample sizes. In one
reasonable setting we find this to reduce from 61\% inconclusiveness to sub 2\%
requiring a sample size on average 22\% larger. 
We also provide an online calculator with source code which aims to make usage of
“Precision is the Goal'' more commonplace in planning of data collection as well as
experiment interpretation.
