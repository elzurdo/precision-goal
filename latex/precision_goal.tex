\documentclass{article}
% bibliography setup
\usepackage{natbib}
\bibliographystyle{apalike}  % abbrvnat
\setcitestyle{authoryear,open={(},close={)}} %Citation-related commands
% makes color citations
\usepackage[colorlinks=true,urlcolor=blue,citecolor=red,linkcolor=red,bookmarks=true]{hyperref}
% images
\usepackage{graphicx}
\graphicspath{ {./images/} }

% hypter parameter setup
\usepackage{hyperref}


\title{Improved Sequential Hypothesis Testing with
``Enhanced Precision Is The Goal"}
\date{\today}
\author{Eyal A. Kazin}

\begin{document}
\maketitle

\input precision_goal_abstract

\includegraphics[scale=0.5]{cherry_posteriors.png}

%\includegraphics[width=1\textwidth]{cherry_posteriors.png}

\begin{figure}[h]
    \centering
    \includegraphics[width=1\textwidth]{cherry_posteriors.png}
    \caption{Beta function posteriors of subsamples of example sequence from Figure REF. Panels are results of different stop/decision criteria being triggered. Shaded areas are 95\% HDIs. ROPE within vertical dashed lines.
    Top - HDI+ ROPE triggers when the 95\% HDI is fully outside the ROPE (iteration 126). Incorrectly rejects $\theta_{\rm null}$. Middle - “Precision is the Goal” triggers when 95\% HDI width reaches precision goal of 80\% ROPE width (iteration 598). HDI straddles the ROPE → inconclusive.
    Bottom - “Enhanced Precision is the Goal” triggers when 95\% HDI fully within ROPE and obtains same precision goal (iteration 804). Correctly accepts $\theta_{\rm null}$}
\end{figure}

\section{Introduction}
\input precision_goal_introduction

\subsection{Contributions}
\section{Methods}

\subsection{One Decision Criterion And Three Stop Criteria}
\subsubsection{Region of Practical Equivalence and High Density Interval}

\subsubsection{Decision Criterion: Is The HDI In Or Out Of The ROPE?}

\subsubsection{HDI + ROPE: Location, Location, Location}

\subsubsection{Precision Is The Goal: Width Then Location}

\subsubsection{Enhanced Precision Is The Goal: Width \& Location}

\subsection{Synthetic Data Analysis}


\section{Results}

\section{Discussion}

\section{Conclusion}

\section{Useful Equations}
\input precision_goal_useful_equations

\section{Submission Instructions}

Methods Paper—maximum word count: 6500
Methods Papers present an advance likely to make major impact in one or more applied
fields. While theory (e.g. theorems concerning statistical or
learning-theoretic properties) are welcome, this is not essential, but intuition and
understanding of why the method works is important.
We are also very open to theory in a broader sense including examples and conjectures.
Generally, we would expect empirical results to demonstrate effectiveness.
However, in cases where the extent of methodological/conceptual innovation is large
enough, results themselves may be illustrative and need not be of immediate high impact
in any particular applied domain.

Each piece should include:

Unstructured Abstract—maximum word count: 250
Keywords—maximum of 6 and minimum of 5
May include tables and figures—no limit
May include the following back matter sections:
Acknowledgements
Author contributions (with CRediT details)
Conflicts of interest
Funding
Must include the following:
Data availability
References—no limit
Each submission must contain the following sections and use these terms as the first
level section headers: Introduction, Methods, Results, Discussion and Conclusion
(Discussion and Conclusion may be combined).

{\bf Resources}

\href{https://academic.oup.com/rssdat/pages/general-instructions}{RSS Instructions} 

\href{https://academic.oup.com/pages/authoring/books/preparing-your-manuscript/working-in-latex}{RSS Working in \LaTeX}.

\bibliography{references.bib}

\end{document}