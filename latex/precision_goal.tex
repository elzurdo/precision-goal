\documentclass{article}
% use for page size and margins: https://www.overleaf.com/learn/latex/Page_size_and_margins
\usepackage[a4paper, margin=0.7in]{geometry}
% bibliography setup
\usepackage{natbib}
\bibliographystyle{apalike}  % abbrvnat
\setcitestyle{authoryear,open={(},close={)}} %Citation-related commands
% makes color citations
\usepackage[colorlinks=true,urlcolor=blue,citecolor=red,linkcolor=red,bookmarks=true]{hyperref}
% images
\usepackage{graphicx}
\graphicspath{ {./images/} }

% hypter parameter setup
\usepackage{hyperref}
% for table setup
\usepackage{multirow}

% for pseudo code
\usepackage{algorithm}
\usepackage{algpseudocode}

% to highlight lines of code, as per: https://tex.stackexchange.com/questions/149779/how-can-i-colourfuly-highlight-some-lines-of-an-algorithm-using-algorithm2e
\usepackage{xcolor}
\def\HiLi{\leavevmode\rlap{\hbox to \hsize{\color{yellow!50}\leaders\hrule height .8\baselineskip depth .5ex\hfill}}}


\title{Improved Sequential Hypothesis Testing with
``Enhanced Precision Is The Goal"}
\date{\today}
\author{Eyal A. Kazin}

\begin{document}
\maketitle

\input precision_goal_abstract

\section{Introduction}
\input precision_goal_introduction

\subsection{Contributions}

\subsubsection{Paper Structure}
The paper is organized as follows.
In the Methods section we describe three (four if Frequentist?)
stop criteria as well as the setup to test on synthetic deichotomous data.
In the Results section we present the results of the synthetic data analysis as well as analytical analysis.
In the Discussion section we discuss the implications of the results (Reword this).
In the Conclusion section we summarize the findings and suggest future directions (Reword this).


\section{Methods}
\input precision_goal_methods


\section{Results}
\input precision_goal_results


\section{Discussion}

Aspects not addressed in the paper:

\begin{itemize}
  \item adding a prior (quite trivial for this setup -- just updating success and failure counts)
  \item continuous data
  \item non analytiucal solutions - would require sampling methods like MCMC which would take much longer to calculate (may be less of an issue with future computer power, e.g, quantum computers)
\end{itemize}

\section{Conclusion}

\section{Useful Equations}
\input precision_goal_useful_equations

\section{Submission Instructions}

Methods Paper—maximum word count: 6500
Methods Papers present an advance likely to make major impact in one or more applied
fields. While theory (e.g. theorems concerning statistical or
learning-theoretic properties) are welcome, this is not essential, but intuition and
understanding of why the method works is important.
We are also very open to theory in a broader sense including examples and conjectures.
Generally, we would expect empirical results to demonstrate effectiveness.
However, in cases where the extent of methodological/conceptual innovation is large
enough, results themselves may be illustrative and need not be of immediate high impact
in any particular applied domain.

Each piece should include:

Unstructured Abstract—maximum word count: 250
Keywords—maximum of 6 and minimum of 5
May include tables and figures—no limit
May include the following back matter sections:
Acknowledgements
Author contributions (with CRediT details)
Conflicts of interest
Funding
Must include the following:
Data availability
References—no limit
Each submission must contain the following sections and use these terms as the first
level section headers: Introduction, Methods, Results, Discussion and Conclusion
(Discussion and Conclusion may be combined).

{\bf Resources}

\href{https://academic.oup.com/rssdat/pages/general-instructions}{RSS Instructions} 

\href{https://academic.oup.com/pages/authoring/books/preparing-your-manuscript/working-in-latex}{RSS Working in \LaTeX}.

\bibliography{references.bib}

\end{document}