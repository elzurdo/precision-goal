\documentclass{article}
% use for page size and margins: https://www.overleaf.com/learn/latex/Page_size_and_margins
\usepackage[a4paper, margin=0.7in]{geometry}
% bibliography setup
\usepackage{natbib}
\bibliographystyle{apalike}  % abbrvnat
\setcitestyle{authoryear,open={(},close={)}} %Citation-related commands
% makes color citations
\usepackage[colorlinks=true,urlcolor=blue,citecolor=red,linkcolor=red,bookmarks=true]{hyperref}
% images
\usepackage{graphicx}
\graphicspath{ {./images/} }

% hypter parameter setup
\usepackage{hyperref}
% for table setup
\usepackage{multirow}

% for pseudo code
\usepackage{algorithm}
\usepackage{algpseudocode}

% to highlight lines of code, as per: https://tex.stackexchange.com/questions/149779/how-can-i-colourfuly-highlight-some-lines-of-an-algorithm-using-algorithm2e
\usepackage{xcolor}
\def\HiLi{\leavevmode\rlap{\hbox to \hsize{\color{yellow!50}\leaders\hrule height .8\baselineskip depth .5ex\hfill}}}


\title{Improved Sequential Hypothesis Testing with
``Enhanced Precision Is The Goal"}
\date{\today}
\author{Eyal A. Kazin}

\begin{document}
\maketitle

\input precision_goal_abstract

\section{Introduction}
\input precision_goal_introduction


\section{Contributions}

This paper makes the following contributions:
\begin{itemize}
  \item \textbf{Identifies a Limitation of Precision-Based Stopping Rules:} We show that the ``Precision is the Goal'' (PitG) method, while unbiased, can result in a high rate of inconclusive outcomes, particularly when the null hypothesis is true.
  \item \textbf{Proposes an Enhanced Stopping Algorithm:} We introduce the ``Enhanced Precision is the Goal'' (ePitG) method, which requires both the precision and decision criteria to be met simultaneously. This approach substantially reduces the rate of inconclusive results with only a moderate increase in sample size.
  \item \textbf{Provides Analytical and Empirical Evaluation:} We analytically derive and empirically validate the performance of ePitG and compare it to existing methods using synthetic dichotomous data.
  \item \textbf{Offers Practical Guidance and Tools:} We provide pseudo-code, Python implementations, and an online calculator to facilitate adoption of the method in practical settings.
  \item \textbf{Demonstrates Broader Applicability:} While our focus is on dichotomous data, the approach is general and can be extended to other data types and experimental designs.
\end{itemize}

\subsection{Paper Structure}
The paper is organized as follows. In the Methods section, we describe three stop criteria as well as the setup to test on synthetic dichotomous data. In the Results section, we present the results of the synthetic data analysis as well as analytical analysis. In the Discussion section, we discuss the implications of the results. In the Conclusion section, we summarize the findings and suggest future directions.


\section{Methods}
\input precision_goal_methods


\section{Results}
\input precision_goal_results


\section{Discussion}

Aspects not addressed in the paper:

\begin{itemize}
  \item adding a prior (quite trivial for this setup -- just updating success and failure counts)
  \item continuous data
  \item non analytiucal solutions - would require sampling methods like MCMC which would take much longer to calculate (may be less of an issue with future computer power, e.g, quantum computers)
\end{itemize}

\section{Conclusion}

\section{Useful Equations}
\input precision_goal_useful_equations

\section{Submission Instructions}

Methods Paper—maximum word count: 6500
Methods Papers present an advance likely to make major impact in one or more applied
fields. While theory (e.g. theorems concerning statistical or
learning-theoretic properties) are welcome, this is not essential, but intuition and
understanding of why the method works is important.
We are also very open to theory in a broader sense including examples and conjectures.
Generally, we would expect empirical results to demonstrate effectiveness.
However, in cases where the extent of methodological/conceptual innovation is large
enough, results themselves may be illustrative and need not be of immediate high impact
in any particular applied domain.

Each piece should include:

Unstructured Abstract—maximum word count: 250
Keywords—maximum of 6 and minimum of 5
May include tables and figures—no limit
May include the following back matter sections:
Acknowledgements
Author contributions (with CRediT details)
Conflicts of interest
Funding
Must include the following:
Data availability
References—no limit
Each submission must contain the following sections and use these terms as the first
level section headers: Introduction, Methods, Results, Discussion and Conclusion
(Discussion and Conclusion may be combined).

{\bf Resources}

\href{https://academic.oup.com/rssdat/pages/general-instructions}{RSS Instructions} 

\href{https://academic.oup.com/pages/authoring/books/preparing-your-manuscript/working-in-latex}{RSS Working in \LaTeX}.

{\bf Talking Points to Include}

* Frequentist methods cannot accept null hypothesis:  In the Frequentist case practitioners set the stop criterion that triggers according to a sample to hypothesis p-value and then decide if to reject the null hypothesis (Frequentist methods cannot accept null hypotheses because p-values do not contain enough information).


\bibliography{references.bib}

\end{document}