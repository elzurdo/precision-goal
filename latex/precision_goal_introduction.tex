
Empirical hypothesis testing is foundational to scientific progress, enabling researchers to assess how well a sample represents an underlying truth. A central challenge in this process is determining when enough data has been collected to make a reliable decision, especially as the accuracy of statistical assessments depends heavily on sample size.  Careful planning for statistical power and precision is essential to avoid biased inference and ensure meaningful results.

Sequential hypothesis testing offers a flexible alternative to fixed-sample designs by allowing data to be evaluated as it accumulates. This approach is widely used in fields such as clinical trials, quality control, A/B testing, and financial monitoring, offering potential benefits in efficiency, timeliness, and ethical considerations. Central to this approach is the distinction between two critical components: the \textit{stopping rule}, which dictates when data collection ends, and the \textit{decision rule}, which determines the final verdict on the hypothesis (e.g., accept, reject, or inconclusive). While these components are conceptually distinct, many operational methods conflate them, triggering a stop exactly when a decision criterion is met.

This simultaneous application of stopping and decision rules is common in widely used Frequentist and Bayesian methods—such as p-value thresholds or posterior-based heuristics (e.g., HDI+ROPE, Bayes Factors). However, if not carefully designed, this coupling can lead to confirmation bias, especially when stopping is triggered by extreme or unrepresentative samples early in the sequence. This phenomenon, known as early peaking, can result in premature and biased conclusions.

To address this, \cite{kruschke2015doing} advocated for a method that strictly decouples these two steps. He proposed using a predetermined target for posterior precision as the sole \textit{stopping rule}, independent of the hypothesis outcome. Only once this precision goal is met is the \textit{decision rule} applied to accept or reject the null hypothesis (the importance of precision as a goal was also pointed out by \citealt{maxwell1984}). \cite{kruschke2015doing} demonstrated that this ``Precision is the Goal'' (PitG) method eliminates bias in outcome sampling. However, our analysis reveals a subtle but important limitation: when the null hypothesis is true, PitG often yields a high rate of inconclusive results, rather than correctly accepting the null.

While scientific incentives often favor the discovery of new effects, the rigorous confirmation of the null hypothesis is equally critical in fields ranging from bioequivalence trials for generic drugs \citep{karalis2012} to ``do-no-harm'' regression testing in software deployment and determining the futility of expensive public policy interventions. In this paper, we propose an enhanced approach—``Enhanced Precision is the Goal'' (ePitG)—which recouples the stopping and decision rules in a conservative manner. Specifically, ePitG requires that \textit{both} the precision goal (stopping rule) and a conclusive hypothesis outcome (decision rule) be met simultaneously to cease data collection.

Examining on similar dichotomous data as in the \cite{kruschke2015doing} setup, this more conservative rule substantially reduces the rate of inconclusive outcomes (e.g., from ~60\% to $<$2\% in our experiments), with only a moderate increase in average sample size. We demonstrate that ePitG is especially effective when the null hypothesis is true, while remaining nearly indistinguishable from PitG in scenarios where the null can be strongly rejected.

Our results, based on simulations with dichotomous data, provide practical guidance for researchers seeking to balance efficiency, bias, and conclusiveness in sequential hypothesis testing. We also provide analytical insights, code, and tools to facilitate adoption of the method in applied settings.

