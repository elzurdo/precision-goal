\section*{Appendix A: Detailed Binomial Experiments}\label{sec:binomial_appendix}

We start by examining in detail simulation results for three scenarios:
\begin{enumerate}
  \item \textbf{Fair Coin ($\theta_{\rm true}=0.5$)}: The null hypothesis is true.
  \item \textbf{Highly Loaded Coin ($\theta_{\rm true}=0.6$)}: A strong effect exists.
  \item \textbf{Loaded Coin ($\theta_{\rm true}=0.57$)}: A reasonable effect exists near the ROPE boundary, challenging the sensitivity of the algorithms.
\end{enumerate}

\subsection{Loaded Coin ($\theta_{\rm true}=0.6$)}\label{sec:loaded_coin}

This experiment represents a case where the null hypothesis is false and there is a fairly strong effect. Results for these experiments are displayed in Figures \ref{fig:loaded0pt6_iter_vs_rate} and \ref{fig:loaded0pt6_decisions}, as well as Table \ref{tab:loaded0pt6_iter_vs_rate}, which are similar to those presented for the fair coin case in Section \ref{sec:fair_coin}.

We observe that for all three algorithms, there is a 0\% False Negative Rate (see Accept column in Table \ref{tab:loaded0pt6_iter_vs_rate}); none of the experiments incorrectly accepted the null hypothesis.

The \textit{HDI+ROPE} algorithm, however, is highly biased, with a mean success rate at stopping of $\overline{\theta} \approx 0.644$. This bias arises because many experiments stop very early, resulting in a mean stop iteration of $\overline{n} \approx 268$ (with a large standard deviation of $\sigma_n=291$), which is less than half the expected precision goal value of $N_{\theta_{\rm true}}=575.2$.

Both precision-based algorithms yield unbiased outcomes. However, \textit{Precision is the Goal} (PitG) remains somewhat indecisive, with 29.2\% of experiments ending as inconclusive. In contrast, \textit{Enhanced Precision is the Goal} (ePitG) is significantly more decisive, clarifying the outcome in 99.8\% of cases (compared to 98.4\% for the fair coin). In this scenario, ePitG requires only 10.7\% more samples than PitG on average.

The symmetry observed in the fair coin case (Figure \ref{fig:fair_iter_vs_rate}) around $\theta_{\rm true}=0.5$ is, as expected, absent here. In Figure \ref{fig:loaded0pt6_iter_vs_rate}, this manifests in the sharp slope of the PitG results (blue circles), which asymptote towards the precision goal value of $N_{\theta_{\rm true}}=575.2$ (dashed vertical line) from above ($\hat{\theta} \ge \theta_{\rm true}$) and not from the bottom.
That said, symmetry of this problem dicates that results for a loaded coine with value $\theta_{\rm true}=0.4$ would be the mirror image of those for $\theta_{\rm true}=0.6$.

\begin{figure}[h!]
  \centering
  \includegraphics[width=1\textwidth]{loaded0pt6_experiments_iter_vs_rate.png}
  \caption{Outcomes of 500 experiments with loaded coin $\theta_{\rm true}=0.6$. Layout and symbols follow Figure \ref{fig:fair_iter_vs_rate}.}
  \label{fig:loaded0pt6_iter_vs_rate}
\end{figure}

\begin{table}[h!]\label{tab:loaded0pt6_iter_vs_rate}
  \begin{center}
  \begin{tabular}{c|c|c|c|c|c|c|c|c}
    \hline
    Algorithm & Accept & Reject & Conclusive & Inconclusive & $\overline{n}$ & $\sigma_n$ & $\overline{\theta}$ & $\sigma_{\hat{\theta}}$\\
    \hline
    HDI+ROPE & 0.0	& 0.998	& 0.998 &	0.002	& 268.3 &	290.8 & 0.6441 &	0.0539 \\
    PitG & 0.0 &	0.708 &	0.708 &	0.292	& 573.7	& 10.2 &	0.6011 &	0.02038 \\
    ePitG & 0.0	& 0.998	& 0.998	& 0.002	& 634.9	& 149	& 0.6034	 & 0.0174 \\
    \hline
  \end{tabular}
  \caption{Statistical summaries of 500 loaded coin experiments ($\theta_{\rm true}=0.6$). Compare with the fair coin baseline in Table \ref{tab:fair_iter_vs_rate}.
  Note: \textit{Accept} + \textit{Reject} = \textit{Conclusive}; \textit{Conclusive} + \textit{Inconclusive} = 1.}
\end{center}
\end{table}

\begin{figure}[h!]
  \centering
  \includegraphics[width=1\textwidth]{loaded0pt6_experiment_decision_rates.png}
  \caption{Cumulative distribution of decisions for $\theta_{\rm true}=0.6$. Compare with Figure \ref{fig:fair_decisions}. Note the high decisiveness of ePitG (green, right panel) compared to PitG (blue, right panel) and the early stopping bias of HDI+ROPE (red, left panel).}
  \label{fig:loaded0pt6_decisions}
\end{figure}

\subsection{Close to the ROPE ($\theta_{\rm true}=0.53, 0.57$)}

We now explore the cases of $\theta_{\rm true}=0.57$ and $0.53$
where we expect a lot of inconclusive outcomes due to straddlers of the ROPE.


\begin{figure}[h!]
  \centering
  \includegraphics[width=1\textwidth]{loaded0pt57_experiments_iter_vs_rate.png}
  \caption{Similar to Figure \ref{fig:fair_iter_vs_rate} but with
  loaded coin $\theta_{\rm true}=0.57$.
  Small symbols are individual experiment outcomes. Large are experiments
  mean values. When using HDI+ROPE as the stop criterion results in the red squares.
  Precision is the Goal as the criterion results in the blue circles
  and the Enhanced PitG as the green Xs. $\theta_{\rm true}$ is the solid horizontal line and
  the ROPE is the horizontaldashed lines. Summary stats are in Table \ref{tab:fair_iter_vs_rate}. TK: analytical results
  }
  \label{fig:loaded0pt57_iter_vs_rate}
\end{figure}


\begin{table}[h!]\label{tab:loaded0pt57_iter_vs_rate}
  \begin{center}
  \begin{tabular}{c|c|c|c|c|c|c|c|c}
    \hline
    Algorithm & Accept & Reject & Conclusive & Inconclusive & $\overline{n}$ & $\sigma_n$ & $\overline{\theta}$ & $\sigma_{\hat{\theta}}$\\
    \hline
    HDI+ROPE & 0.0080	& 0.648 &	0.656 &	0.344 &	804.6	& 611.0	& 0.605	& 0.058 \\
    PitG & 0.0005 &	0.157 &	0.158 &	0.843	& 586.5	& 7.1 &	0.570 &	0.020\\
    ePitG & 0.0035 &	0.542 &	0.545 & 	0.455 &	1127.2 &	395.9	& 0.576 &	0.017  \\
    \hline
  \end{tabular}
  \caption{Similar to Table \ref{tab:fair_iter_vs_rate} but for $\theta_{\rm true}=0.57$ Statistic summaries of 500 experiments of the three stop criteria shown in
  Figure \ref{fig:fair_iter_vs_rate}. {\it Accept}
  is the fraction of experiments which results in acceptence of $\theta_{\rm null}$,
  and similar in reverse for {\it Reject}. {\it Conclusive} is the sum of Accept
  and Reject and {\it Inconclusive} is its complementary.
  The mean stop iteration is $\overline{n}$ and the standard deviation $\sigma_n$
  The mean sample rate when the stop is triggered is $\overline{\theta}$ and its standard deviation is $\sigma_{\hat{\theta}}$.
  }
\end{center}
\end{table}

% hdi_rope	0.0080	& 0.6480 &	0.6560 &	0.3440 &	804.6165	& 611.021115	& 0.604697	& 0.058295
% pitg	0.0005 &	0.1570 &	0.1575 &	0.8425	& 586.4470	& 7.116144 &	0.570288 &	0.020308
% epitg	0.0035 &	0.5415 &	0.5450 & 	0.4550 &	1127.1755 &	395.941462	& 0.575583 &	0.017217


\begin{figure}[h!]
  \centering
  \includegraphics[width=1\textwidth]{loaded0pt57_experiment_decision_rates.png}
  \caption{Similar information as in Figure \ref{fig:fair_iter_vs_rate} but focusing on
  the distribution of decisions. Left panel is for HDI+ROPE. Right panel are both
  precision based methods. Horizontal axis- iteration. Vertical axis- proportion of
  decisions made up to (and including) each iteration. The sum of Reject (red dashed),
  Accept (green solid) and Inconclusive/Collect-more- (gray dot-dashed) decisions at
  each iteration is 100\%.
  }
  \label{fig:loaded0pt57_decisions}
\end{figure}
